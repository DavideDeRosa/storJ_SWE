\documentclass{article}

\usepackage[utf8]{inputenc}
\usepackage[T1]{fontenc}
\usepackage{lipsum}
\usepackage{graphicx}
\usepackage{amsmath}
\usepackage[margin=1in]{geometry}
\usepackage{titlesec}
\usepackage{enumitem}
\usepackage{geometry}
\usepackage{tabularx}
\usepackage{caption}
\usepackage{fixltx2e}
\usepackage{booktabs}
\usepackage{float}  
\usepackage{graphicx}
\usepackage{floatflt,epsfig}
\usepackage[margin=1in]{geometry} 
\usepackage{lipsum}
\usepackage{graphicx}
\usepackage{listings}
\usepackage{xcolor}
\usepackage{color}

\lstdefinelanguage{Java}{
  keywords={abstract,assert,boolean,break,byte,case,catch,char,class,const,continue,default,do,double,else,enum,extends,false,final,finally,float,for,goto,if,implements,import,instanceof,int,interface,long,native,new,null,package,private,protected,public,return,short,static,strictfp,super,switch,synchronized,this,throw,throws,transient,true,try,void,volatile,while},
  morekeywords={[2]System,out},
  morecomment=[l]{//},
  morecomment=[s]{/*}{*/},
  morestring=[b]",
  basicstyle=\small\ttfamily,
  keywordstyle=\color{blue}\bfseries,
  keywordstyle={[2]\color{orange}\bfseries},
  commentstyle=\color{green!70!black},
  stringstyle=\color{red},
  showstringspaces=false,
  tabsize=2,
  breaklines=true,
  breakatwhitespace=true,
  frame=single, 
  captionpos=b
}

\titleformat{\section}[block]
{\large\bfseries}{\thesection}{1em}{}

\titleformat{\subsection}[block]
{\normalfont\large\bfseries}{\thesubsection}{1em}{}

\begin{document}

\pagestyle{empty}

\begin{titlepage} 
\begin{center}
    {{\Large{\textsc{Alma Mater Studiorum - Università di Bologna}}}}
    \rule[0.1cm]{\textwidth}{0.1px}
    \rule[0.5cm]{\textwidth}{0.6px}\\
    {\large{SCUOLA DI SCIENZE \\ Corso di Laurea in Informatica per il Management}}
\end{center}

\vspace{50px}

\begin{center}
    {\LARGE{{\bf Piattaforma ESQL}}}\\
\end{center}

\vspace{115px}
\par
\noindent
\begin{minipage}[t]{0.04\textwidth}
~
\end{minipage}
\begin{minipage}[t]{0.4\textwidth}
\end{minipage}
\hfill
\begin{minipage}[t]{0.4\textwidth}\raggedleft
    {\fontsize{12}{13}{DOCUMENTAZIONE SVOLTA DA:}\\
\fontsize{12}{13}{\it Canghiari Matteo \\ De Rosa Davide \\ Ghazanfar Tabish \\ Nadifi Ossama}}
\end{minipage}
\begin{minipage}[t]{0.04\textwidth}
~
\end{minipage}

\vspace*{210px}

\begin{center}
    \large{Anno Accademico 2023/2024}
\end{center}
\end{titlepage}

\section{Design Model}

\subsection{Modello di dominio} 
\large

\subsection{Diagramma dei casi d'uso}
\large
\textbf{METTERE DIAGRAMMA}\vspace*{7pt}\\

\newpage
\textbf{Registrazione}
\begin{itemize}[label = { }]
    \itemsep0px
    \item ID: UC1 - Registrazione
    \item Attori: Utente
    \item Precondizioni: 
        \begin{itemize}[label = {-}]
            \item Non presente
        \end{itemize}
    \item Main sequence: 
        \begin{enumerate}
            \item Utente inserisce e invia i dati
            \item Sistema acquisisce e salva i dati
            \item Sistema invia esito positivo
            \item Utente visualizza esito positivo
        \end{enumerate}
    \item Alternative sequence:
        \begin{enumerate}
            \item Utente inserisce e invia i dati non idonei
            \item Sistema acquisisce e individua un errore
            \item Sistema invia esito negativo
            \item Utente visualizza esito negativo
        \end{enumerate}
    \item Postcondizioni: 
        \begin{itemize}[label = {-}]
            \item Sistema salva i dati nella piattaforma
        \end{itemize}
\end{itemize}
\textbf{Login}
\begin{itemize}[label = { }]
    \itemsep0px
    \item ID: UC2 - Login
    \item Attori: Utente
    \item Precondizioni: 
        \begin{itemize}[label = {-}]
            \item Utente ha effettuato la registrazione
        \end{itemize}
    \item Main sequence: 
        \begin{enumerate}
            \item Utente inserisce e invia i dati
            \item Sistema acquisisce e confronta con i dati salvati
            \item Sistema nega l'accesso e invia esito positivo
            \item Utente visualizza esito positivo
        \end{enumerate}
    \item Alternative sequence:
        \begin{enumerate}
            \item Utente inserisce e invia i dati non idonei
            \item Sistema acquisisce e individua un errore
            \item Sistema invia esito negativo
            \item Utente visualizza esito negativo
        \end{enumerate}
    \item Postcondizioni: 
        \begin{itemize}[label = {-}]
            \item Utente autenticato dalla piatttaforma
        \end{itemize}
\end{itemize}
\textbf{Logout}
\begin{itemize}[label = { }]
    \itemsep0px
    \item ID: UC3 - Logout
    \item Attori: Utente
    \item Precondizioni: 
        \begin{itemize}[label = {-}]
            \item Utente autenticato dalla piattaforma
        \end{itemize}
    \item Main sequence: 
        \begin{enumerate}
            \item Utente effettua la selezione di logout
            \item Sistema registra lo stato Utente
        \end{enumerate}
    \item Alternative sequence:
        \begin{enumerate}
            \item Non presente
        \end{enumerate}
    \item Postcondizioni: 
        \begin{itemize}[label = {-}]
            \item Utente disconesso dalla piatttaforma
        \end{itemize}
\end{itemize}
\textbf{Pagamento}
\begin{itemize}[label = { }]
    \itemsep0px
    \item ID: UC4 - Pagamento
    \item Attori: Utente
    \item Precondizioni: 
        \begin{itemize}[label = {-}]
            \item Utente autenticato dalla piattaforma
            \item Utente non abbia effettuato alcun pagamento
        \end{itemize}
    \item Main sequence: 
        \begin{enumerate}
            \item Utente inserisce e invia i dati
            \item Sistema acquisisce i dati
            \item Sistema processa pagamento e invia esito positivo
            \item Utente visualizza esito positivo
        \end{enumerate}
    \item Alternative sequence:
        \begin{enumerate}
            \item Utente inserisce e invia i dati
            \item Sistema acquisisce i dati
            \item Sistema processa pagamento e invia esito negativo
            \item Utente visualizza esito negativo
        \end{enumerate}
    \item Postcondizioni: 
        \begin{itemize}[label = {-}]
            \item Sistema salva lo stato Utente di pagamento
        \end{itemize}
\end{itemize}
\textbf{Creazione scenario}
\begin{itemize}[label = { }]
    \itemsep0px
    \item ID: UC5 - Creazione scenario
    \item Attori: Utente
    \item Precondizioni: 
        \begin{itemize}[label = {-}]
            \item Utente autenticato dalla piattaforma
        \end{itemize}
    \item Main sequence: 
        \begin{enumerate}
            \item Utente inserisce e invia i dati
            \item Sistema acquisisce i dati
            \item Sistema controlla i dati e invia esito positivo
            \item Utente visualizza esito positivo
        \end{enumerate}
    \item Alternative sequence:
        \begin{enumerate}
            \item Utente inserisce e invia i dati non completi
            \item Sistema acquisisce i dati
            \item Sistema controlla i dati e invia esito negativo
            \item Utente visualizza esito negativo
        \end{enumerate}
    \item Postcondizioni: 
        \begin{itemize}[label = {-}]
            \item Sistema salva i dati Scenario
        \end{itemize}
\end{itemize}
\textbf{Creazione scelta}
\begin{itemize}[label = { }]
    \itemsep0px
    \item ID: UC6 - Creazione scelta
    \item Attori: Utente
    \item Precondizioni: 
        \begin{itemize}[label = {-}]
            \item Utente autenticato dalla piattaforma
            \item Utente abbia creato componenti Scenario
        \end{itemize}
    \item Main sequence: 
        \begin{enumerate}
            \item Utente inserisce e invia i dati
            \item Sistema acquisisce i dati
            \item Sistema controlla i dati e invia esito positivo
            \item Utente visualizza esito positivo
        \end{enumerate}
    \item Alternative sequence:
        \begin{enumerate}
            \item Utente inserisce e invia i dati non completi
            \item Sistema acquisisce i dati
            \item Sistema controlla i dati e invia esito negativo
            \item Utente visualizza esito negativo
        \end{enumerate}
    \item Postcondizioni: 
        \begin{itemize}[label = {-}]
            \item Sistema salva i dati Scelta
        \end{itemize}
\end{itemize}
\textbf{Salvataggio storia}
\begin{itemize}[label = { }]
    \itemsep0px
    \item ID: UC7 - Salvataggio storia
    \item Attori: Utente
    \item Precondizioni: 
        \begin{itemize}[label = {-}]
            \item Utente autenticato dalla piattaforma
            \item Utente abbia creato componenti Scenario
            \item Utente abbia creato componenti Scelta
        \end{itemize}
    \item Main sequence: 
        \begin{enumerate}
            \item Utente richiede salvataggio Storia
            \item Sistema acquisisce i dati
            \item Sistema controlla i parametri di salvataggio
            \item Sistema invia esito positivo
            \item Utente visualizza esito positivo
        \end{enumerate}
    \item Alternative sequence:
        \begin{enumerate}
            \item Utente richiede salvataggio Storia
            \item Sistema acquisisce i dati
            \item Sistema controlla i parametri di salvataggio
            \item Sistema invia esito negativo
            \item Utente visualizza esito negativo
        \end{enumerate}
    \item Postcondizioni: 
        \begin{itemize}[label = {-}]
            \item Sistema salva i dati Storia
            \item Sistema rende componente Storia giocabile
        \end{itemize}
\end{itemize}
\textbf{Gestione storia}
\begin{itemize}[label = { }]
    \itemsep0px
    \item ID: UC8 - Gestione storia
    \item Attori: Utente
    \item Precondizioni: 
        \begin{itemize}[label = {-}]
            \item Utente autenticato dalla piattaforma
        \end{itemize}
    \item Main sequence: 
        \begin{enumerate}
            \item Utente richiede rimozione della Storia
            \item Sistema riceve richiesta
        \end{enumerate}
    \item Alternative sequence:
        \begin{enumerate}
            \item Non presente
        \end{enumerate}
    \item Postcondizioni: 
        \begin{itemize}[label = {-}]
            \item Sistema elimina componente Storia
        \end{itemize}
\end{itemize}
\textbf{Gestione scenario}
\begin{itemize}[label = { }]
    \itemsep0px
    \item ID: UC9 - Gestione scenario
    \item Attori: Utente
    \item Precondizioni: 
        \begin{itemize}[label = {-}]
            \item Utente autenticato dalla piattaforma
        \end{itemize}
    \item Main sequence: 
        \begin{enumerate}
            \item Utente richiede rimozione dello Scenario
            \item Sistema riceve richiesta
        \end{enumerate}
    \item Alternative sequence:
        \begin{enumerate}
            \item Utente richiede modifica testuale dello Scenario
            \item Sistema riceve richiesta
            \item Utente inserisci i dati sostitutivi
            \item Sistema acquisisce modifiche
        \end{enumerate}
    \item Postcondizioni: 
        \begin{itemize}[label = {-}]
            \item Sistema elimina componente Scenario
            \item Sistema modifica Scenario
        \end{itemize}
\end{itemize}
\textbf{Gestione scelta}
\begin{itemize}[label = { }]
    \itemsep0px
    \item ID: UC10 - Gestione scelta
    \item Attori: Utente
    \item Precondizioni: 
        \begin{itemize}[label = {-}]
            \item Utente autenticato dalla piattaforma
        \end{itemize}
    \item Main sequence: 
        \begin{enumerate}
            \item Utente richiede rimozione della Scelta
            \item Sistema riceve richiesta
        \end{enumerate}
    \item Alternative sequence:
        \begin{enumerate}
            \item Utente richiede modifica testuale della Scelta
            \item Sistema riceve richiesta
            \item Utente inserisci i dati sostitutivi
            \item Sistema acquisisce modifiche
        \end{enumerate}
    \item Postcondizioni: 
        \begin{itemize}[label = {-}]
            \item Sistema elimina componente Scelta
            \item Sistema modifica Scelta
        \end{itemize}
\end{itemize}
\textbf{Gestione storia}
\begin{itemize}[label = { }]
    \itemsep0px
    \item ID: UC11 - Gestione storia
    \item Attori: Utente
    \item Precondizioni: 
        \begin{itemize}[label = {-}]
            \item Utente autenticato dalla piattaforma
            \item Utente abbia effettuato un pagamento
            \item Storia deve essere giocabile
        \end{itemize}
    \item Main sequence: 
        \begin{enumerate}
            \item Utente seleziona una Storia giocabile
            \item Sistema mostra Scenario composto dalle sue Scelte
            \item Utente indica Scelta
            \item Sistema acquisisce la Scelta
            \item Sistema mostra Scenario successivo
        \end{enumerate}
    \item Alternative sequence:
        \begin{enumerate}
            \item Utente seleziona una Storia giocabile
            \item Sistema mostra Scenario composto dalle sue Scelte
            \item Utente indica Scelta
            \item Sistema acquisisce la Scelta
            \item Sistema mostra Scenario finale
        \end{enumerate}
    \item Postcondizioni: 
        \begin{itemize}[label = {-}]
            \item Sistema salva la Partita
        \end{itemize}
\end{itemize}
\textbf{Gestione partita}
\begin{itemize}[label = { }]
    \itemsep0px
    \item ID: UC12 - Gestione partita
    \item Attori: Utente
    \item Precondizioni: 
        \begin{itemize}[label = {-}]
            \item Utente autenticato dalla piattaforma
            \item Utente abbia effettuato un pagamento
            \item Storia deve essere giocabile
        \end{itemize}
    \item Main sequence: 
        \begin{enumerate}
            \item Utente richiede rimozione Partita inerente alla Storia
            \item Sistema riceve richiesta
        \end{enumerate}
    \item Alternative sequence:
        \begin{enumerate}
            \item Utente seleziona una Partita da riprendere
            \item Sistema mostra Scenario composto dalle sue Scelte
            \item Utente indica Scelta
            \item Sistema acquisisce la Scelta
            \item Sistema mostra Scenario Successivo
        \end{enumerate}
    \item Postcondizioni: 
        \begin{itemize}[label = {-}]
            \item Sistema rimuove la Partita
            \item Utente riprende la Partita
        \end{itemize}
\end{itemize}
\textbf{Creazione oggetto}
\begin{itemize}[label = {-}]
    \itemsep0px
    \item ID: UC13 - Creazione oggetto
    \item Attori: Utente
    \item Precondizioni: 
        \begin{itemize}[label = {-}]
            \item Utente autenticato dalla piattaforma
        \end{itemize}
    \item Main sequence: 
        \begin{enumerate}
            \item Utente inserisce e invia i dati
            \item Sistema acquisisce i dati
            \item Sistema controlla i dati e invia esito positivo
            \item Utente visualizza esito positivo
        \end{enumerate}
    \item Alternative sequence:
        \begin{enumerate}
            \item Utente inserisce e invia i dati non completi
            \item Sistema acquisisce i dati
            \item Sistema controlla i dati e invia esito negativo
            \item Utente visualizza esito negativo
        \end{enumerate}
    \item Postcondizioni: 
        \begin{itemize}[label = {-}]
            \item Sistema salva i dati dell'Oggetto
        \end{itemize}
\end{itemize}
\textbf{Gestione oggetto}
\begin{itemize}[label = {-}]
    \itemsep0px
    \item ID: UC14 - Gestione oggetto
    \item Attori: Utente
    \item Precondizioni: 
        \begin{itemize}[label = {-}]
            \item Utente autenticato dalla piattaforma
        \end{itemize}
    \item Main sequence: 
        \begin{enumerate}
            \item Utente richiede rimozione dell'Oggetto
            \item Sistema riceve richiesta
        \end{enumerate}
    \item Alternative sequence:
        \begin{enumerate}
            \item Utente richiede modifica testuale dell'Oggetto
            \item Sistema riceve richiesta
            \item Utente inserisce modifiche
            \item Sistema acquisisce modifiche
        \end{enumerate}
    \item Postcondizioni: 
        \begin{itemize}[label = {-}]
            \item Sistema rimuove componente Oggetto
            \item Sistema modifica componente Oggetto
        \end{itemize}
\end{itemize}

\newpage
\subsection{Burndown chart}

\newpage
\section{Manuale dell'utente}
\subsection*{Home}
Sezione a cui si accede avviando l'applicativo. All'interno di questa è possibile ottenere una breve descrizione del progetto, visualizzare le specifiche e la repository GitHub, ed infine accedere alla relazione. Dalla navbar si può accedere alle schermate di \textit{Accedi} e \textit{Registrati}.
\textbf{METTERE FOTO}

\subsection*{Registrazione}
Sezione che permette di registrarsi alla piattaforma. Offre la possibilità di accedere alla schermata di \textit{Accedi} sia tramite la schermata principale che tramite navbar.
\textbf{METTERE FOTO}

\subsection*{Accedi}
Sezione che permette di autenticarsi alla piattaforma. Offre la possibilità di accedere alla schermata di \textit{Registrazione} sia tramite la schermata principale che tramite navbar.
\textbf{METTERE FOTO}

\subsection*{Homepage storJ}
Sezione a cui si accede in seguito all'autenticazione. Permette di visualizzare una breve descrizione delle azioni eseguibili. Attraverso la navbar si potranno raggiungere le schermate relative a \textit{Paga}, \textit{Crea}, ed infine effettuare il \textit{Logout}.
\textbf{METTERE FOTO}
Nel caso in cui pagamento sia già stato effettuato, la voce \textit{Paga} sarà sostituita da \textit{Gioca}.

\subsection*{Paga}
Sezione che permette di effettuare il pagamento, in modo da poter sbloccare tutte le funzionalità della piattaforma. Sarà necessario inserire l'ammontare del pagamento, titolare e numero della carta, ed infine il relativo CVV. Potrebbe essere necessario effettuare più tentativi per far sì che vada a buon fine.
\textbf{METTERE FOTO}

\subsection*{Storie}
Sezione che consente la gestione delle storie. Permette di visualizzare le storie create in precedenza. Ad ognuna di esse sono assegnati due bottoni, necessari per la \textit{modifica} e l'\textit{eliminazione}. È presente una navbar contenente il collegamento alla schermata dedicata alla \textit{creazione} delle storie. Nel caso la storia sia stata salvata, la \textit{modifica} permetterà solo di cambiare i testi all'interno di questa.
\textbf{METTERE FOTO}

\subsection*{Form creazione storia}
Sezione che permette la \textit{creazione} di una storia. È necessario inserire il titolo e la categoria.
\textbf{METTERE FOTO}

\subsection*{Scenari}
Sezione che consente la \textit{gestione} di una storia specifica. Permette di visualizzare gli scenari appartenenti alla storia. Ad ognuno di essi sono associate le relative informazioni e due bottoni, utilizzati per l'accesso alla \textit{gestione} delle scelte e per la \textit{cancellazione} dello scenario. È presente una navbar che permette di salvare la storia e accedere alle sezioni relative alla \textit{creazione} di uno scenario, e infine alla \textit{gestione} degli oggetti.
\textbf{METTERE FOTO}
Una volta effettuato il salvataggio della storia, sarà possibile solo modificare i testi relativi ad essa.
\textbf{METTERE FOTO}

\subsection*{Form creazione scenario}
Sezione che permette la \textit{creazione} di uno scenario. È necessario inserire i dati relativi ad esso, quindi il testo, la tipologia di scenario, la tipologia di risposta ed infine indicare la presenza o meno del drop di un oggetto.\vspace*{7pt}\\
\textit{Nota bene}: è necessario aver già creato l'oggetto che si vuole far rilasciare dallo scenario.
\textbf{METTERE FOTO}

\subsection*{Indovinello}
Sezione che permette la \textit{gestione} della risposta ad un indovinello. È possibile visualizzare i dati relativi a quest'ultimo, quindi il testo, la risposta associata ed infine gli scenari destinazione in seguito alla risposta corretta piuttosto che alla risposta errata. 
\textbf{METTERE FOTO}
È possibile cancellare l'indovinello, se presente, e attraverso navbar ci sarà la possibilità di crearne uno nuovo.
\textbf{METTERE FOTO}

\subsection*{Form creazione indovinello}
Sezione che permette la \textit{creazione} della risposta ad un indovinello. È necessario inserire il testo del quesito, la risposta attesa ed infine gli scenari di destinazione in seguito alla risposta corretta piuttosto che alla risposta errata. 
\textbf{METTERE FOTO}

\subsection*{Scelta multipla}
Sezione che permette la \textit{gestione} delle risposte ad una domanda a scelta multipla. È possibile visualizzare i dati relativi alle opzioni già presenti, quindi il testo, lo scenario successivo alla risposta, e la presenza o meno di un oggetto richiesto.\vspace*{7pt}\\
\textit{Nota bene}: è necessario aver già creato l'oggetto richiesto.\vspace*{7pt}\\
È possibile inoltre cancellare le opzioni, se presenti, e attraverso navbar ci sarà la possibilità di crearne nuove.
\textbf{METTERE FOTO}

\subsection*{Form creazione multipla}
Sezione che permette la \textit{creazione} della risposta ad una domanda a scelta multipla. È necessario inserire il testo del quesito, lo scenario successivo alla risposta, e la presenza o meno di un oggetto richiesto. 
\textbf{METTERE FOTO}

\subsection*{Oggetti}
Sezione che permette la \textit{gestione} degli oggetti di una storia. È possibile visualizzare i dati relativi ad ognuno, quindi il titolo e la descrizione, ed un bottone che permette la cancellazione. Attraverso navbar ci sarà la possibilità di creare nuovi oggetti.
\textbf{METTERE FOTO}

\subsection*{Form creazione oggetti}
Sezione che permette la \textit{creazione} di un oggetto. È necessario inserire i dati relativi ad esso, quindi il titolo e la descrizione.
\textbf{METTERE FOTO}

\subsection*{Storie giocabili}
Sezione che permette la \textit{selezione} della storia da giocare. È possibile visualizzare tutte le storie giocabili con le varie informazioni, permettendo la ricerca attraverso dei filtri. 
\textbf{METTERE FOTO}\vspace*{7pt}\\
Per ogni storia presente ci sarà la possibilità di \textit{avviare} la partita, o in alternativa di \textit{visualizzare} l'\textit{anteprima} di essa.

\subsection*{Gioca}
Sezione in cui avviene lo \textit{svolgimento} della partita. All'interno della schermata sarà presente il testo dello scenario ed in base alla tipologia della domanda verranno visualizzati i bottoni contenenti le opzioni di risposta piuttosto che l'area testuale per inserire la risposta all'indovinello.
\textbf{METTERE FOTO}\vspace*{7pt}\\
\textbf{METTERE FOTO}\vspace*{7pt}\\
Attraverso la barra di navigazione sarà sempre possibile consultare l'\textit{inventario} della partita.
\textbf{METTERE FOTO}\vspace*{7pt}\\

\subsection*{Partite giocate}
Sezione in cui sono presenti le partite avviate dall'utente. Per ogni partita sarà possibile riprendere la storia dallo scenario in cui era stata interrotta oppure rimuovere i dati di gioco relativi ad essa.
\textbf{METTERE FOTO}\vspace*{7pt}\\

\clearpage
\section*{Manuale dello sviluppatore}

\subsection{Setup e deploy}
Unico requisito per il deploy dell'applicativo è \textbf{Docker}. Una volta scaricata la \textit{repository} da \textit{GitHub}, sarà sufficiente eseguire all'interno della cartella il comando \textbf{docker-compose up --build}; quest ultimo permetterà a \textit{Docker} di eseguire la fase di \textit{build} e successivamente di \textit{avviare} i \textit{container} relativi ai diversi servizi che compongono l'applicativo.\vspace*{7pt}\\
Per terminare l'esecuzione, eseguire all'interno della cartella precedente il comando \textbf{docker-compose down}, il quale terminerà ed eliminerà i \textit{container} dell'applicazione.\vspace*{7pt}\\
È stato scelto di utilizzare \textit{Docker} e \textit{Docker-Compose} per il deploy dell'applicativo per avere una maggiore facilità di \textit{deploy} e per permettere l'utilizzo del servizio senza dover installare in locale alcuna dipendenza software, come \textit{Maven}, \textit{NodeJS} o \textit{PostgreSQL}.\vspace*{7pt}\\
È possibile osservare all'interno della repository i \textit{Dockerfile} inerenti ai vari servizi, i quali permettono a \textit{Docker} di avviare i \textit{container}. Per ottenere un avvio simultaneo di tutti i servizi viene utilizzato un file \textit{docker-compose.yml}, il quale permette l'avvio dell'applicativo con un singolo comando.\vspace*{7pt}\\
Un esempio di \textit{Dockerfile} appartiene allo \textit{Spring Boot} del backend, il quale si suddivide in due fasi:
\vspace*{7pt}
\begin{lstlisting}[language = JAVA]
FROM maven:3.8.3-openjdk-17 AS spring-builder
WORKDIR /usr/src/app
COPY . /usr/src/app
RUN mvn clean package

FROM eclipse-temurin:17
WORKDIR /opt/app
COPY --from=spring-builder /usr/src/app/target/storj-1.0.0.jar /opt/app/storj.jar
EXPOSE 8080
CMD ["java", "-jar", "/opt/app/storj.jar"]
\end{lstlisting}
La prima fase esegue il comando \textit{Maven} per ottenere il file \textit{.jar} eseguibile. La seconda avvia un \textit{container} eseguendo l'applicativo di backend.\vspace*{7pt}\\
Passando invece al \textit{docker-compose.yml}, la suddivisione è netta tra i servizi:\vspace*{7pt}
\begin{lstlisting}[language = JAVA]
version: '3'

services:
  db:
    image: 'postgres:16'
    environment:
      - POSTGRES_USER=postgres
      - POSTGRES_PASSWORD=root
      - POSTGRES_DB=storj
    container_name: db
    networks:
      - storjnetwork
    volumes:
      - ./db/volumes/data:/var/lib/postgresql/data
      - ./db/sql/:/docker-entrypoint-initdb.d/

aymentservice:
    build: payment
    container_name: paymentservice
    depends_on:
    - app
    networks:
    - storjnetwork
    ports:
    - "6789:6789"

app:
    build: backend
    container_name: storj
    depends_on:
    - db
    networks:
    - storjnetwork
    ports:
    - "8080:8080"

angular:
    build: frontend
    container_name: angular
    depends_on:
    - app
    networks:
    - storjnetwork
    ports:
    - "4201:4200"

networks:
    storjnetwork:
      driver: bridge
\end{lstlisting}
Come si può notare, vengono avviati quattro servizi, distinti in:
\begin{itemize}[label = {-}]
    \itemsep0em
    \item \textbf{Database}, servizio inerente al \textit{DBMS} scelto per la persistenza dei dati; in questa casistica ricade in \textit{PostgreSQL}. Vengono quindi configurati alcuni valori di \textit{environment} per permettere la creazione del database all'interno del \textit{container}. Vengono inoltre utilizzati i \textit{volumi} per garantire la persistenza dei dati e per creare lo schema del database al primo avvio del \textit{container}
    \item \textbf{Paymentservice}, servizio di pagamento esterno richiesto dalla traccia. Viene avviato un \textit{container Docker} che avvia l'eseguibile fornito nelle specifiche, esponendo la chiamata \textit{API} per il pagamento
    \item \textbf{App}, backend dell'applicativo, realizzato con \textit{Spring Boot}. L'\textit{image} relativa al suo \textit{container} viene realizzata attraverso il \textit{Dockerfile} osservato in precedenza
    \item \textbf{Angular}, frontend dell'applicativo, realizzato con \textit{Angular}. Come avviene per il backend, l'\textit{image} relativa al suo \textit{container} viene realizzata attraverso un \textit{Dockerfile}, presente nella cartella del frontend
\end{itemize}
Si può notare infine l'utilizzo dei networks di \textit{Docker}, i quali permettono ai diversi container di comunicare tra loro. Vengono inoltre esposte diverse porte per permettere l'accesso ai servizi offerti dai \textit{container} sulla \textit{macchina host}.

\subsection{Frontend}
Per il frontend è stato utilizzato il framework \textbf{Angular}, in grado di garantire un approccio orientato a facilitare lo sviluppo delle interfacce grafiche. \textit{Angular} fornisce un elevato livello di \textit{reusability} del codice, grazie alla suddivisione dell'architettura che lo contraddistingue in differenti \textit{componenti}; inoltre il framework è caratterizzato da un'ulteriore capacità, la quale consiste in azioni automatiche di \textit{sincronizzazione} tra la \textit{User Interface} e il modello dati. Infine, uno dei fattori principali è dettato dalla \textit{Dependency Injection}, poiché facilita la manipolazione dei legami posti tra \textit{componenti} e \textit{servizi} che compongono l'applicazione.\vspace*{7pt}\\
Come è stato già citato prima, il progetto è stato concepito seguendo un'architettura modulare, che divide l'applicazione in \textit{componenti} e \textit{servizi}. Questo approccio favorisce la separazione delle responsabilità, migliorando la manutenibilità e la scalabilità del codice. Tuttavia, è bene individuare quali siano i punti salienti su cui occorre soffermarsi, descritti come segue:
\begin{itemize}[label = {-}]
    \itemsep0em
    \item \textbf{Componenti}, rappresentano le parti visive dell'applicazione, ognuna incaricata di gestire una specifica funzionalità o area dell'interfaccia utente. Nell'applicazione sono presenti tre elementi fondamentali, suddivisi in: 
    \begin{itemize}[label = {-}]
        \itemsep0em
        \item \textbf{Form}, attuati per consentire all'utente di creare oppure modificare i vari componenti delle proprie storie
        \item \textbf{Navbar}, elemento imprescindibile per la navigazione tra le pagine che compongono il sito; dinamica rispetto all'interazioni precedenti effettuate dall'utente
        \item \textbf{Pop-up}, entità grafica visualizzata esclusivamente all'interno della fase di gioco, utilizzata per mostrare informazioni inerenti alla storia selezionata
    \end{itemize}
    Oltre alle tre citate, sono molteplici le \textit{pagine} presenti, le quali non sono contraddistinte da mansioni specifiche ma possiedono un ruolo fondamentale, poiché illustrano tutti i \textit{componenti} definiti dai vari utenti
    \item \textbf{Servizi}, tramite per acquisire i dati dal backend, posti all'interno del database e comunicati dalle \textit{API} (per ulteriori approfondimenti consultare sezione backend). Fungono da intermediari, sovrapposti tra i componenti e le \textit{API}, affinché i dati possano essere forniti e successivamente visualizzati a schermo
    \item \textbf{LocalStorage}, strumento versatile, inerente non solo a fasi di sviluppo, principalmente attuato per azioni di \textit{debugging}, ma anche a funzionalità mirate alla persistenza dei dati, garantendo un'esperienza all’utente fluida e consistente. Uno degli utilizzi principi riguarda il controllo dell'autenticazione dell'utente, necessario per un corretto funzionamento delle guardie
    \item \textbf{Guardie}, funzionalità built-in del framework, stabilite per circoscrivere il reindirizzamento tra le pagine. Nell'applicazione sono presenti due tipologie di \textit{guardie}, in cui la prima, denominata \textbf{AccessGuard}, definisce se l'utente sia autenticato o meno; mentre la seconda, chiamata \textbf{PaymentAccessGuard}, impedisce all'utente di accedere al form \textit{payment-page}, qualora abbia già effettuato un pagamento andato a buon fine
    \item \textbf{Classi}, utilizzate per definire l'architettura interna dei vari oggetti manipolati dall'app, inerenti alla struttura del database di riferimento (ad esempio scenari, storie oppure partite)
\end{itemize}

\subsection*{Backend}
Per il backend è stato utilizzato il framework \textbf{Spring}, principalmente attuato per velocizzare le fasi di sviluppo. \textit{Spring} inoltre fornisce funzionalità molto comode per lo sviluppo di backend, come la facilità di sviluppo di \textbf{REST API} ed il collegamento ad un \textbf{database}. Inoltre, garantisce l'approccio relativo alla \textbf{Dependency Injection}, attraverso il quale il contenitore \textit{Spring} “inietta” oggetti in altre “dipendenze”. In poche parole, ciò consente un accoppiamento libero dei componenti e sposta la responsabilità della gestione dei componenti al framework.\vspace*{7pt}\\
Passando all'implementazione delle \textit{API}, l'approccio utilizzato è \textit{Contract-First} secondo lo standard \textbf{OpenAPI 3.0}. Viene realizzata un'interfaccia, la quale permette ad un plugin di \textit{Spring} di generare automaticamente i diversi metodi che verranno sovrascritti per implementare le funzionalità volute. Sono state realizzate le \textbf{CRUD} di tutte le strutture collegate al \textit{database}, con alcune chiamate \textit{custom} per ottenere liste personalizzate da visualizzare nel frontend.\vspace*{7pt}\\
Il collegamento con il \textit{database} è stato realizzato attraverso il plugin \textbf{Spring Data JPA}, il quale permette in maniera veloce di collegare il backend con la base di dati.\vspace*{7pt}\\
L'architettura delle chiamate API è la seguente:
\begin{itemize}[label = {-}]
    \itemsep0em
    \item Ogni entità della base di dati possiede una corrispettiva classe \textbf{Entity}, che rappresenta la riga all'interno del database come un oggetto \textit{Java}. Per comunicare direttamente col \textit{database}, viene creata un'interfaccia \textbf{Repository} per ogni entità, dove vengono specificate le diverse Query
    \item Attraverso il plugin di \textit{OpenAPI} viene generata per ogni entità un \textbf{Model}, il quale indica la struttura con la quale frontend e backend comunicano
    \item Per passare da \textit{Model} ad \textit{Entity}, e viceversa, viene utilizzato il plugin \textbf{MapStruct}, che permette di passare da un \textit{Model} ad un \textbf{DTO} (\textbf{Data Transfer Object}) e da \textit{DTO} ad \textit{Entity}, e viceversa. Separando \textit{Model} ed \textit{Entity} otteniamo un codice più pulito, non utilizzando nella \textit{business logic} del backend alcun \textit{Model}
    \item I diversi metodi della \textit{Repository} vengono \textit{wrappati} da una classe \textbf{Service}, che riprende tutte le possibili chiamate al database per quella entità
    \item I metodi di ogni \textit{Service} vengono ripresi da una classe di \textbf{Business Logic}, dove tutta la logica viene implementata
    \item In conclusione, i metodi di ogni \textit{Business Logic} vengono richiamati dai \textbf{Controller}, ovvero le implementazioni delle interfacce generate dallo standard \textit{OpenAPI}. All'interno dei controller non c'è alcuna logica, delegata alle classi di \textit{Business Logic}
\end{itemize}
L'architettura delle \textit{API} si basa sul \textbf{Single Responsibility Principle}, il quale afferma che ogni elemento di un programma (classe, metodo oppure variabile) deve avere una sola responsabilità, e che tale responsabilità debba essere interamente incapsulata dall'elemento stesso.\vspace*{7pt}\\
Infine, per la gestione degli errori possibili durante le chiamate \textit{API} è stato implementato un \textit{ExceptionHandlingController}, permettendo una gestione comoda e dinamica delle eccezioni.\vspace*{7pt}\\
Si osserva ora la sezione dei \textbf{test unitari}. Per quanto riguarda la base di dati, è stato utilizzato il plugin \textbf{H2}, creando un \textbf{database in-memory} per eseguire tutti i test. Sono stati realizzati \textit{126 test unitari}, ottenendo un \textbf{code coverage} del \textit{97\%}. I test sono stati effettuati sulle due parti principali del backend, i \textbf{servizi} (che comprendono le classi di \textit{Business Logic} e di \textit{Service}, testano quindi tutta la logica di business e le query) ed i \textbf{mapper}. È stato inoltre utilizzato il plugin \textbf{Jacoco} per ottenere un semplice report:
\textbf{METTERE LA FOTO}\vspace*{7pt}\\
\textit{Nota bene}: non è stato possibile ottenere un \textit{code coverage} maggiore a causa della \textit{randomicità} del pagamento, vincolando i test inerenti a quella parte.

\subsection{Database}
Per la persistenza dei dati della piattaforma è stato utilizzato un \textit{Database Relazionale}. Nel nostro caso abbiamo scelto \textbf{PostgreSQL}. La struttura segue di pari passo le \textit{API} descritte nella parte di backend.  Sono presenti le seguenti tabelle:
\begin{itemize}[label = {-}]
    \itemsep0em
    \item \textbf{utente}, dati inerenti all'utente, come \textit{username}, \textit{password} e se è stato effettuato o meno il \textit{pagamento}
    \item \textbf{storia}, dati inerenti alla storia, come \textit{creatore}, \textit{titolo}, \textit{categoria}, \textit{numero scenari} e se la storia è stata \textit{salvata} come partita giocabile
    \item \textbf{scenario}, dati inerenti allo scenario, come la \textit{storia} a cui appartiene, \textit{testo}, \textit{tipologia} della \textit{risposta} (\textit{Indovinello} oppure \textit{Multipla}) e \textit{tipologia} dello \textit{scenario} (\textit{Iniziale}, \textit{Normale} oppure \textit{Finale})
    \item \textbf{oggetto}, dati inerenti ad un oggetto, come la \textit{storia} a cui \textit{appartiene}, \textit{nome} e \textit{descrizione}
    \item \textbf{drop}, struttura di supporto, la quale permette di comprendere l’oggetto rilasciato all’interno di uno scenario. Al suo interno troviamo lo \textit{scenario} e l'\textit{oggetto} collegati
    \item \textbf{multipla}, dati inerenti alla singola scelta appartenente ad un insieme di scelte di uno scenario. Uno scenario può avere scelte multiple soltanto se appartiene alla tipologia \textit{Multipla}. Possono esserci un minimo di due scelte per scenario, senza limiti superiori. Al suo interno si trova lo \textit{scenario} a cui appartiene, \textit{testo} della scelta e lo \textit{scenario} di \textit{destinazione}
    \item \textbf{required}, struttura di supporto, attuata per definire quale sia l'oggetto richiesto per effettuare una specifica scelta multipla. Composto da una \textit{scelta} e dall'\textit{oggetto} a cui è collegata
    \item \textbf{indovinello}, dati inerenti all'indovinello appartenente ad uno scenario, necessariamente di tipologia \textit{Indovinello}. Può esserci soltanto un indovinello per scenario. A sua volta è composto dallo \textit{scenario} a cui appartiene, da un \textit{testo}, da una \textit{risposta corretta}, da uno \textit{scenario corretto}, qualora la risposta data sia conforme con la domanda, e da uno \textit{scenario errato}, qualora l'utente dovesse indicare una risposta sbagliata
    \item \textbf{partita}, dati inerenti alla partita, come la \textit{storia} a cui si riferisce, l'\textit{utente} che la stia svolgendo e lo \textit{scenario corrente}
    \item \textbf{inventario}, struttura di supporto, la quale permette di comprendere la lista di oggetti appartenenti ad una partita. Definito da una \textit{partita} e da un \textit{oggetto}
\end{itemize}
Il database viene avviato tramite il \textit{docker-compose.yml} utilizzato per avviare tutti i servizi della piattaforma. Vengono inoltre utilizzati i \textit{volumi} per garantire la persistenza dei dati e per creare lo schema del database al primo avvio del \textit{container}. Per quest'ultima, si sfrutta il \textit{docker-entrypoint-initdb.d}, il quale avvia qualsiasi \textit{script SQL} nella directory selezionata. Questo permette la creazione dinamica del \textit{volume} del database al primo avvio del \textit{container}. 

\subsection{Payment}
Per il servizio di pagamento è stato utilizzato l'applicativo fornito nelle specifiche tecniche, il quale espone una \textit{REST API} che simula il funzionamento di un pagamento. La chiamata \textit{API} è stata \textit{wrappata} all'interno delle chiamate del backend, permettendo una gestione più flessibile delle risposte ricevute. L’eseguibile viene eseguito grazie ad un Dockerfile, il quale permette di effettuare il \textit{build} della sua \textit{Docker image} e di essere eseguito con \textit{docker-compose}.

\subsection{DevOps}
Per quanto riguarda la parte di \textbf{Continuous Integration} (\textit{CI}) e di \textbf{Continuous Deployment} (\textit{CD}) sono state automatizzate le \textit{pipeline} attraverso \textbf{Jenkins}. L'esecuzione di \textit{Jenkins} avviene all'interno di una \textit{Virtual Machine} (Ubuntu Server).\vspace*{7pt}\\

\newpage
\subsection{Tavola media dei volumi}
In questa sezione è specificato il numero stimato di istanze per ogni entità e relazione dello schema. I valori sono necessariamente approssimati, ma oltre tutto indicativi. Si prende come riferimento una realtà universitaria.


\begin{table}[h]
    \centering
    \begin{tabularx}{\textwidth}{|X|X|X|X|}
        \hline
        \bf Concetto & \bf Tipo & \bf Volume \\
        \hline
        Utente & Entità & 305 \\
        \hline
        Studente & Entità & 300 \\
        \hline
        Docente & Entità & 5 \\
        \hline
        Tabella\_Esercizio & Entità & 50 \\
        \hline
        Attributo & Entità & 200 \\
        \hline
        Test & Entità & 10 \\
        \hline
        Quesito & Entità & 100 \\
        \hline
        Domanda\_Chiusa & Entità & 50 \\
        \hline
        Opzione\_Risposta & Entità & 150 \\
        \hline
        Domanda\_Codice & Entità & 50 \\
        \hline
        Sketch\_Codice & Entità & 50 \\
        \hline
        Messaggio & Entità & 610 \\
        \hline
        Messaggio\_Studente & Entità & 600 \\
        \hline
        Creazione & Relazione & 50 \\
        \hline
        Completamento & Relazione & 3000 \\
        \hline
        Invio & Relazione & 18000 \\
        \hline
        Ricezione & Relazione & 10 \\
        \hline
        Pubblicazione & Relazione & 6100 \\
        \hline
        Risposta & Relazione & 30000 \\
        \hline
        Composizione & Relazione & 1000 \\
        \hline
        Afferenza & Relazione & 5000 \\
        \hline
        Combinazione & Relazione & 10000 \\
        \hline
        Vincolo Integrità & Relazione & 40000 \\
        \hline
        Disposizione & Relazione & 7500 \\
        \hline
        Soluzione & Relazione & 2500 \\
        \hline
    \end{tabularx}
    \caption{Stima della tavola media dei volumi riferita al progetto svolto.}
\end{table}

\end{document}